\section{Guide to Creating Simulations using the IPS}
\label{sec:simulations}

\par
Once you have all of the components and drivers that you want to use in 
order, you will need to construct a configuration file, build and run the 
IPS.  This guide will explain the elements of the configuration, and brief 
instructions on how to build and run the IPS with your simulation.

\subsection{Configuration: Elements of a Simulation}
\label{sec:config}
\par 
A simulation in the {IPS} is defined by the contents of the configuration 
file.  It is in the configuration files that all of the components, 
locations of data and source files, and other options are set.  There are 
seven sections of the simulation configuration file.  They are briefly 
described below and annotated in the sample directory.

\subsubsection{Paths and Run ID}
\par 
In this section the paths to the \texttt{IPS\_ROOT} (framework source tree), 
\texttt{SIM\_ROOT} (where you want the results to go) and the name of the run are 
specified.  Some other variables relating to these items are also set for 
convenience.  It is highly recommended that you change the name of the 
\texttt{SIM\_ROOT} for each run so as to not clobber your previous results.

\subsubsection{Plasma State Configuration}
\par 
In this section the plasma state working directory is specified as the 
place for plasma state files to be recorded.  The names of the files that 
are going to be used as the plasma state are also recorded here.

\subsubsection{Portal and Logging}
\par 
In this section variables used by the web portal specific to this 
simulation are set, as well as the simulation log level and location.  The 
log level controls how much debugging information is provided by the 
framework.  Each component can set their own log levels in the component 
configuration section.

\subsubsection{Port Configuration}
\par 
This section contains the names of the ports that will be used by the 
driver, along with the implementations.  Ports are similar to the class 
names of the components, they refer to the functionality that the component 
implements.  The port names are how the driver accesses the reference to 
the component.  Be sure these names match what the driver is expecting.

\subsubsection{Component Configuration}
\par 
This section describes how each component should be run.  The class, 
subclass and name are used to create the IPS name of the component and the 
directory structure.  \texttt{NPROC} is the number of processes the binary needs to 
run.  \texttt{BIN\_PATH} is the path to the component script.  Input and output files  
are also listed here.  Any files not listed will not be captured by the 
data management system.

\subsubsection{Time Loop Configuration}
\par 
This section describes how the simulation level time loop is 
constructed.  There are two modes, \texttt{REGULAR} (a list of values from start to 
finish (inclusive) with the specified interval are created) and \texttt{EXPLICIT} (a 
list of times (space separated) is given explicitly by the user).


\subsection{Platform Configuration}
\label{sec:plat_config}
\par
In addition to the simulation configuration file, a platform configuration 
file must be provided to the {IPS} at launch time.  The platform 
configuration file tends to remain unchanged per platform.  There are 
examples in the top level of the {IPS} and in the sample directory.
\par
Advanced users may use the \texttt{NODES} and \texttt{CORES\_PER\_NODE} entries to specify the node and core counts available to the simulation for arbitrary machines.  Users should note that the IPS will not check these values against what the machine reports.


\subsection{Building the IPS}
\label{sec:build}
\par
Before building and running the {IPS}, you should read the \texttt{README} file in the 
top level of the {IPS} tree.  It contains important information about 
dependencies and the directory structure.
\par
You need to source \texttt{swim.bashrc.$\ast$} (the star depends
on your platform, for example, franklin) to set
up variables and load modules.  Now you are ready to build and install the IPS.
\par
These three steps are performed by the following commands in the top level of the IPS tree:
\begin{verbatim}
  . swim.bashrc.<platform>
  make
  make install
\end{verbatim}


\subsection{Running the IPS}
\label{sec:run}
\par
The IPS has been ported to the Cray XT4/XT5 machines jaguar (at NCCS) and franklin (at NERSC), as well as viz/mhd at PPPL (a shared memory machine).  The IPS will run on arbitrary machines if the number of nodes available in the allocation are specified in the platform configuration file.  If no resource information is obtained  the IPS will assume there is only on node with one core available and will run everything on one core serially.  Running the IPS in serial mode is useful for testing the component and services interactions.

\par
To run the IPS several command line arguments must be provided as such:
\begin{verbatim}
ips [--config=CONFIG_FILE_NAME]+ --platform=PLATFORM_FILE_NAME     \
     --log=LOG_FILE_NAME [--debug] [--ftb]
  
Arguments:
  --config
    A simulation configuration file.  There can be multiple simulations 
    executing but each must be described in separate, uniquely named
    configuration files.  At least one must be provided.
  --platform
    The platform file that matches the platform you are using.
  --log
    The name of the log file containing all the debugging output.
  --debug
    Flag to turn on debugging output.  Optional.
  --ftb
    Flag to turn on FTB notifications.  Optional, experimental, only 
    available on jaguar.

\end{verbatim}
where, \texttt{ips} is the IPS executable.

\par
Use \texttt{create\_batch\_script.py} (located in \texttt{IPS\_ROOT/framework/src/}) to create a 
batch script to submit on a particular machine.  It does some error 
checking.  There is a sample batch script in the sample directory if you 
wish to create and edit your own manually.

\begin{verbatim}
python create_batch_script.py --ips=IPS_EXECUTABLE             \
                             [--config=CONFIG_FILE_NAME]+      \
                              --platform=PLATFORM_FILE_NAME    \
                             [--account=CHRGE_ACCOUNT]         \
                             [--queue=BATCH_QUEUE]             \
                             [--walltime=ALLOCATION_TIME]      \
                             [--nproc=NPROCESSES]              \
                             [--debug]                         \
                             [--ftb]                           \
                             [--output=BATCH_SCRIPT]           

Arguments:
  --ips
     This is the IPS executable using the full path.
  --config
     A simulation configuration file.  There can be multiple simulations 
     executing but each must be described in separate, uniquely named 
     configuration files.  At least one must be provided.
  --platform
     The platform file that matches the platform you are using.
  --account
     The account that will be charged for time on the machine.
  --queue
     The queue to which you will submit the job.
  --walltime
     The time to request from the batch scheduler.
  --nproc
     The number of processes to request from the batch scheduler.
  --debug
     Flag to turn on debugging output.
  --ftb
     Flag to turn on FTB notifications.  Only available on jaguar.
  --output
     The name of the batch script you want to generate.
     
\end{verbatim}
