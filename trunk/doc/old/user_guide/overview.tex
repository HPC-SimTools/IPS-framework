\section{Overview of the IPS}
\label{sec:overview}

\par The purpose of this document is to acquaint new users to the Integrated 
Plasma Simulator (IPS) a framework for component coupling developed for 
the Center for Simulation of RF Wave Interactions with 
Magnetohydrodynamics (SWIM) project.  The rest of the document is 
organized into the following sections: motivation and purpose, design and 
implementation, what you can do with the IPS, and how to get started.  For 
those just interested in creating components and simulations, we recommend 
skipping the motivation section, skimming the design and implementation 
and reading the last two sections.  For those more interested in the 
computer science and design choices, the motivation and design sections 
will be useful, as well as the publications listed therein.

\par This document is a work in progress and may have varying degrees of 
completion and detail.

\subsection{Motivation and Purpose}
\label{sec:motivation}

\subsection{Design and Implementation}
\label{sec:design}


\subsection{What YOU can do with the IPS}
\label{sec:usecases}

\par The IPS framework provides a flexible environment to couple multiple 
components concurrently and serially.  The framework provides services for  
file-based data management, task coupling, configuration, monitoring and 
resource management within a single batch allocation.  It also provides an 
event service for adding new functionality and communication.

\par The execution model, that separates component method invocations and the 
launch of (parallel) binaries, allows for many different blocking and non-
blocking task coupling scenarios.  Additionally, multiple simulations may 
be executed within the same framework instance, within the same batch 
allocation.  

\subsection{How to get started}
\label{sec:getstarted}

\par The first thing to do after checking out the source code, is to create any 
components or drivers you wish to use in your simulation.  See 
\texttt{components.txt} and the skeleton implementations in the samples directory.  
Details on how to add components and build and run the system are provided 
in \texttt{components.txt} and \texttt{simulations.txt}.  \texttt{services.txt} is to be used as a 
reference for component developers of what services are available and how 
to use them.
