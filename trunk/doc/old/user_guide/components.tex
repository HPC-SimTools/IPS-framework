\section{Guide to IPS Components}
\label{sec:components}

In the IPS, \emph{component} refers to the python script that interacts with the 
framework, services, other components and (optionally) launches binaries to 
perform a specific function for the simulation.  These components tend to 
be physics components that perform a specific modeling function for a 
coupled physics simulation.

This guide will help you construct a component (or driver) to be used in 
various simulations.  You will most likely need to refer to services.txt 
for the services that framework provides for component activities.  Also, 
some examples are provided in the sample directory and can serve as a 
skeleton for implementing your new components!

You will want to put newly created components in their proper directories 
under the \texttt{\$IPS\_ROOT/components/} directory (where \texttt{\$IPS\_ROOT} is your 
checkout of the IPS).  The components directory is organized by class and 
subclass.  For instance, all components that are drivers are in the class 
driver, and the subclass describes the implementation of the driver.  This 
reflects the desire of the SWIM project to collect multiple implementations 
of different types of components.  For instance, \emph{AORSA} and \emph{TORIC} are both 
RF codes, and are located in the \texttt{components/rf/} directory there are 
directories for \texttt{aorsa} and \texttt{toric}, respectively.  In the \texttt{components/rf/
aorsa/} directory the source for the aorsa component resides, along with 
any other supporting files (like a makefile).  The AORSA code resides in 
the \texttt{phys\_bin/} a separate directory that contains the binaries of codes 
being used by SWIM.  This directory lives on the target platforms and 
contains compiled codes that work on that platform as provided by the code 
owner.  See the building and running sections of the \texttt{simulations.txt} file, 
and the \texttt{README} file in the top level of the IPS for more information on 
building and running the IPS and the file structure.

\subsection{Component Object Structure:}
\label{sec:comp_struct}
A component implementation inherits some functionality from the IPS Component 
object to perform some common startup and termination tasks.  Component 
writers do not need to worry about these calls.

These functions should be copied from the skeleton component and not 
modified in most cases:\\

\texttt{\_\_init\_\_(self, services, config)}:
\par
  This function gets and saves the component specific configuration
  information for use by the component instance throughout the simulation,
  and sets up the reference to the services that the component instance can
  use.

\par
These functions contain the physics functionality of the component and 
should be written by the component writer.  Please *augment* the skeleton 
component, as there might be some lines of code that all components need to 
function properly in the system.\\

\texttt{init(self, timestamp)}:
\par Any initial set up for the component is done here.  "timestamp" is the
  simulation timestamp from the driver's time loop.  This function 
  typically gets called at the beginning of the simulation for any 
  pre-simulation processing.  Additional per-step processing may happen in 
  the step function.\\

\texttt{step(self, timestamp)}:
\par
  This is where the component performs its calculations at a particular
  simulation level timestamp.  This function typically gets called during
  each iteration through the time loop. Services related to launching 
  tasks, and data management for the step should be used here.  See the 
  services section for details on how to choose the right function.  Step 
  pre- and post-processing may be done here as well.\\

\texttt{finalize(self, timestamp)}:
\par
  This is where the component performs any clean up after the simulation is
  over.  It typically gets called at the end of the simulation after the 
  simulation has completed.

\subsection{Driver Object Structure:}
\label{sec:driver_struct}

The driver has the same structure as the component with the following
exceptions:
\begin{itemize}
  \item the init happens before the ``simulation'' starts
  \item the step function contains all of the logic for the simulation.  It contains three phases:
    \begin{itemize}
       \item setup:
       \begin{itemize}
         \item component references are obtained and initialized
         \item timeloop is obtained from configuration
       \end{itemize} 
       \item time loop:
       \begin{itemize}
         \item in the time loop, components are called.  The simulation workflow 
          logic is implemented using the different call methods.  See the 
          services section for details.
       \end{itemize}
       \item clean up:
         \begin{itemize}
           \item components are finalized
         \end{itemize}
     \end{itemize}
  \item finalize occurs after all of the components have been finalized.
\end{itemize}

