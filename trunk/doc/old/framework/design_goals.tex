\section{Design Goals}
\label{sec:design_goals}

Design goals were agreed upon at the start of the \ac{swim} project. They include:
\begin{itemizer}
\item{Handle job management and monitoring on different platforms, especially
unique DOE petascale ones}

\item{Data managment, movement, location, and sophisticated query}

\item{Minimal perturbation to ongoing scientific research using the extant
constituent codes and software}

\item{Maximize reliability of the systems codes}

\item{Extensibility, programmability, and longevity}

\item{Leverage existing technology and interoperate with other fusion
simulation efforts}

\end{itemizer}
Implementation choices have been driven by these design goals.
One choice has been to use a primarily {\em scripting} based framework system.
This allows us to use
different \ac{gui} and workflow composition systems while retaining a large base
of intercomponent interactions defined by \ac{swim} runs.  Scripts have had greater
longevity than \ac{gui}s, and some of the scripts used to coordinate and launch HPC
runs in fusion simulations date back decades.

Another principle driven by the goals is the {\em decoupling} of parts of the \ac{swim}
framework as much as possible (this is actually a form of the software component
dictum that required interfaces should be as minimal as possible).
Foremost, no simulation should fail unnecessarily because some other component
failed. A sophisticated data managment system is provided through the portal,
but decoupling means it can be used independently of the portal - and some runs
can be made without using the data manager at all. Of course in that case you
lose the ability to track and automatically archive the files associated with a
run.

Also as much as possible \ac{swim} uses utilities that have proven reliable, are
widely used, and which seem likely to continue to be maintained and developed
by other projects. Included in this list are the Python language, the Gridsphere
portal, and MySQL for data management underpinnings.

As another example, 
using Python (versus, e.g., Java)
for the scripts was chosen because current DOE HPC platforms all support some
form of Python, even if they do not have Java yet. 


